\section{Distributed Objects}
La comunicazione è un aspetto fondamentale dei sistemi distribuiti. \\
Socket e MPI sono meccanisimi di basso livello, scomodi da usare.
Nel tempo sono nati meccanismi di più alto livello come procedure remote (RPC) e oggetti distribuiti, che permettessero una programmazione similare
a quella classica anche su sistemi distribuiti.

Problemi del \textit{parameter passing}:
 - type management: diversi linguaggi di programmazione forniscono tipi di dato diversi e rappresentano tipi simili in modo diverso.
 In più, a parità di linguaggio di programmazione, lo stesso tipo può essere rappresentato diversamente dipendentemente dall'architettura.
 - semantica del passaggio: per valore o per riferimento. Se passo un oggetto per riferimento ma non condivido la memoria su cui è allocato, il passaggio è inutile.

Mettiamo ad esempio un passaggio per riferimento in C, come sappiamo quanto è grande l'area di memoria da passare?
Amesso che usiamo il passaggio per valore, cosa succede se l'oggetto contiene dei puntatori ad altri oggetti? shallow vs deep copy.

Static vs dynamic invocation.
Invocazione statica: oggetto e metodo da invocare sono decisi a compile time, e.g. obj.method(params) \\
Invocazione dinamica: la scelta di oggetto e metodo avvengono a runtime, e.g. invoke(obj, id(method), params) \\

Reliability.
Nel caso remoto possono avvenire errori di comunicazione dovuti alla rete ed errori locali ai due nodi.

Deployment. Aka come distribuire gli oggetti sul sistema distribuito?
Diversi criteri di ottimizzazione:
 - minimizzzazione delle interzzion
 - bialndiamento dle carico
 - allocazione hardwarre aware

I criteri possono essere contrastanti.
La rilocazione a runtime degli oggetti aggiunge complessità.

Esistono diversi middleware, che mettono a disposizione diversi servizi, come: naming e distribuzione (RMI), discovery, persistenza e replica.

Tecnologie omogenee vs eterogenee.
Nelle tecnologie omogenee un solo linugaggio di programmazione è consentito, come Java RMI. Il supporto ai nodi remoti può essere fornito direttamente dal linguaggio. \\
Nelle tecnologie eterogenee sono consentiti più linguaggi di programmazione, come COBRA, ma in questo caso un middleware è necessario.

Distributed vs remote objects. \\
Un oggetto distribuito e1 un oggeto il cui stato (aka le cui proprietà) sono distribuiti tra più nodi. \\
Invece, un oggetto remoto è un oggetto è un oggetto che risiede su un nodo remoto e i cui metodi vengono invocati da un altro nodo.

